\documentclass{amsart}
\usepackage[margin=1in]{geometry}
\usepackage[utf8]{inputenc}

\title[\small{Kac-Moody groups and Cluster Algebras}]
  {AIM SQuaRE Proposal\\ 
  \small{Kac-Moody groups and Cluster Algebras}}

\begin{document}
  \maketitle
  
  \section*{Overview}
    The goal of this AIM SQuaRE Proposal is to complete a project started at the AMS Mathematics Research Community on Cluster Algebras in June 2014.
    Our aim is to realize acyclic cluster algebras with principal coefficients as the ring of coordinates of a reduced double Bruhat cell in the corresponding Kac-Moody group and to describe at least some of the cluster variables in terms of generalized minors.
    
  \subsection*{Participants}
    The participants to the meeting will be 
    \begin{itemize}
      \item Dylan Rupel, Kenna Visiting Assistant Professor, University of Notre Dame 
      \item Salvatore Stella, INdAM-marie Curie fellow, Università di Roma ``La Sapienza''
      \item Harold Williams, NSF Postdoctoral Fellow, University of Texas at Austin
    \end{itemize}
  
  \section*{Project description}

  The main object of study in this project are cluster algebras; these are a class of commutative algebras introduced by Fomin and Zelevinsky as a tool to understand Lusztig's dual canonical bases and  defined via a recursive procedure called mutation. 
  These mutations produce, at the same time, the generators of the algebra (the ``cluster variables'') and the relations among them (the ``exchange relations'').
  The combinatorial backbone of this process can be encoded by a (valued) quiver and many features of the resulting algebras can be read directly from this datum.
  For example cluster algebras can be classified in terms of their ``growth'' rate and the class they belong to depends only on this quiver.
  Incidentally this classification is extremely close to the classification of semisimple Lie algebras and groups by Cartan matrices and Dynkin diagrams. 

  Even though, at a first glance, their definition might look quite technical cluster algebras appear naturally as the ring of coordinates of many notable varieties. 
  One of the earliest families of examples, indeed, came from the work of Berenstein Fomin and Zelevinsky; they realized that the ring of coordinates of any double Bruhat cell of a Lie group $G$ carries a cluster algebra structure. 
  Unfortunately the type of the algebras constructed in this way, in general, will not be related to the type of $G$. 

  Motivated by this mismatch Yang and Zelevinsky realized that among all the double Bruhat cells of $G$ there is a notable subset that give rise to algebras of the same type of $G$. 
  More precisely, for each valued quiver $Q$ whose underlying graph is a Dynkin graph of finite type they found a double Bruhat cell that has $Q$ as the initial quiver. 
  Moreover and arguably more interestingly, after passing to a suitable subvariety, they were able to provide explicit formulas for all the cluster variables in terms of ``generalized minors'' (i.e. certain matrix coefficients of specific representations of $G$).

  Based on our former works the initial goal we had was to extend the results of Yang and Zelevinsky to any Kac-Moody group $G$. 
  In doing so we soon realized that the general case presents a level of complication not appearing in finite type.
  The quickest way to explain it is to consider the representation theory of $Q$.
  It is a well established fact, if $Q$ is acyclic, that there is a bijection (the Caldero-Chapoton map) between the cluster variables in $\mathcal{A}(Q)$ and rigid indecomposable $Q$-modules. 
  These, for a non-finite type $Q$, come naturally in three families: preprojectives, postinjectives, and regulars.
  We observed the same type of trichotomy at the level of generalized minors. 
  Indeed cluster variables seem to appear as matrix coefficients of either highest weight, lowest weight or level zero representations of $G$.  
  While the first two are well understood objects, less is known of level zero representations.
  
  At the moment, for any Kac-Moody group, using some Lewis Carroll type relations we have complete control on both preprojective and  postinjective cluster variables as functions on the appropriate reduced double Bruhat cell.
  For affine types we also have a conjectural understanding of regular cluster variables again as matrix coefficients. 
  During our previous meetings we were able to lay the general framework for our project and to prove completely the results we are after for affine type $A$. 

  Our aim is now to complete the picture for the remaining affine types.
  While in type $A$ it was sufficient to use the Weyl character formula to gain all the informations we need on the level zero representations at hand, more work is needed in the remaining types.
  Indeed to apply our conjectural formulas we need to know the exact structure of the representations and not just the dimensions of their weight spaces. 
  
  The hope is to leverage cluster algebras to get new insights into the level zero representations of affine Kac-Moody groups. 
  Indeed our construction suggests a deep interplay between representations of different kinds, since the exchange relations in the cluster algebra imply certain nontrivial relations between the characters of the corresponding representations.  
  In particular, by relating level zero representations to both highest and lowest weight representations, we seek to gain a deeper understanding of the structure of the former. 
  

  \section*{Previous meetings}
  \begin{itemize}
    \item 
      June 2014: MRC Cluster Algebras
    \item
      February 2015: NCSU University
    \item
      September 2015: Notre Dame University (Sponsored by AMS)
  \end{itemize}

  \begin{thebibliography}{XXX}
    \bibitem{BFZ05}
    A.~Berenstein and S.~Fomin and A.~Zelevinsky, Cluster algebras III: Upper bounds and double Bruhat cells, \textsl{Duke Math. J.} \textbf{126} (2005), no.~1, pp.~1--52.

    \bibitem{Wil13}
    H.~Williams, Cluster ensembles and Kac-Moody groups, \textsl{Adv. Math.}, \text{247}, (2013), pp.~1--40.

    \bibitem{YZ08}
    S.~Yang and A.~Zelevinsky, Cluster algebras of finite type via Coxeter elements and principal minors, \textsl{Transform. Groups}, \textbf{13} (2008), no.~3-4, pp.~855-895.
    
  \end{thebibliography}

\end{document}
